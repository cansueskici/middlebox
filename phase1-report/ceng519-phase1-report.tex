\documentclass[10pt,a4paper, margin=1in]{article}
\usepackage{fullpage}
\usepackage{amsfonts, amsmath, pifont}
\usepackage{amsthm}
\usepackage{graphicx}
\usepackage{subfig}
\usepackage{float}
\usepackage{longtable}
\usepackage{tkz-euclide}
\usepackage{titling}
\usepackage{booktabs}
\usepackage{comment}
\usepackage{tikz}
\usepackage{pgfplots}
\pgfplotsset{compat=1.13}

\setlength{\droptitle}{-7em} 
\title{CENG519 - Phase 1 Report}
\author{
  Cansu Eskici\\
  2588036}
\begin{document}
\maketitle
\section*{Introduction}
For this phase, I have implemented a simple processor in Python, and I conducted the experiment with 5 different delay values. 
In order to have reliable results, each value is obtained from sending the ping packets 50 times. 
On the X-axis of the chart, the numbers represent the negative power of the exponent for the lambda parameter in the random delay function, as $5e^{-x}$. For example, 3 means $5e^{-3}$, while 4 means $5e^{-4}$.


\vspace{1cm}

\begin{minipage}{0.5\textwidth}
    \begin{tikzpicture}
        \centering
        \begin{axis}[
        %title={ping-delay},
            xlabel={Mean Value For Random Delay},
            ylabel={Average RTT},
            ymin= 0, ymax = 27,
            ytick = {0,5,10,15,20,25},
            %gridstyle=dashed,
            ymajorgrids=true,
            ]
        \addplot coordinates {
          (3,23.250)(4,12.363)(5,12.056)(6,9.645)(7,9.230)
        };
      \end{axis}
    \end{tikzpicture}
\end{minipage}
\begin{minipage}{0.5\textwidth}
    \begin{table}[H]
    \centering
    \begin{tabular}{|c|c|}
        \hline
        Mean Delay Values & Mean RTT \\
        \hline
        ${5e^-3}$ & 23.250 \\
        ${5e^-4}$ & 12.363 \\
        ${5e^-5}$ & 12.056 \\
        ${5e^-6}$ & 9.645 \\
        ${5e^-7}$ & 9.230 \\
        \hline
    \end{tabular}
    \caption*{Mean Delay Values and Mean RTTs }
    \label{tab:xy_table}
\end{table}
\end{minipage}
\vspace{0.4cm}
\section*{Results}

The X values in the chart represent the negative power of the exponent in the random delay function,
meaning that as the value on the X axis increases, the mean delay actually decreases.
The table shows the mean delay values and the mean RTTs for each delay value.
Both the chart and the table show that the mean RTT decreases as the negative power of the exponent increases, or as the mean value for the random delay decreases.
The relationship between the mean value for random delay and the average RTT is not directly proportional.
There is a significant drop in average RTT when the negative power is increased from 3 to 4, but after 4, the differences are not as big as that.


\section*{Discussion}
The findings indicate that the average RTT decreases as the mean value for the random delay decreases.
Shorter delays result in shorter RTT times.
This is expected since the lower delays are likely to result in quicker packet processing and transmission.
The non-proportional relationship between the mean value for random delay and the average RTT is probably due to the exponential nature of the random delay function.
Another indicator of this is the almost indistinguishable differences among the mean RTT values with the lower mean delay values.

 \end{document}

